\documentclass[12pt,a4paper]{article}
\usepackage[utf8]{inputenc}
\usepackage[french]{babel}
\usepackage{amsmath}
\usepackage{amssymb}
\usepackage{amsthm}
\usepackage{geometry}
\usepackage{graphicx}
\usepackage{enumitem}
\usepackage{fancyhdr}
\usepackage{xcolor}
\usepackage{listings}
\usepackage{booktabs}
\usepackage{tikz}
\usepackage{float}
\usepackage{algorithm}
\usepackage{algorithmic}

\geometry{margin=2.5cm}
\pagestyle{fancy}
\fancyhf{}
\rhead{Moteur d'Allocation FoodOps}
\lfoot{Documentation Technique}
\rfoot{\thepage}

\definecolor{codeblue}{rgb}{0.13,0.29,0.53}
\definecolor{codegray}{rgb}{0.5,0.5,0.5}
\definecolor{codegreen}{rgb}{0,0.6,0}
\definecolor{lightgray}{rgb}{0.95,0.95,0.95}
\definecolor{darkblue}{rgb}{0.0,0.0,0.6}

\lstset{
    language=Python,
    basicstyle=\ttfamily\small,
    keywordstyle=\color{codeblue},
    commentstyle=\color{codegreen},
    stringstyle=\color{red},
    numbers=left,
    numberstyle=\tiny\color{codegray},
    stepnumber=1,
    numbersep=5pt,
    backgroundcolor=\color{lightgray},
    showspaces=false,
    showstringspaces=false,
    showtabs=false,
    frame=single,
    rulecolor=\color{black},
    tabsize=2,
    captionpos=b,
    breaklines=true,
    breakatwhitespace=false,
    title=\lstname
}

\title{\textbf{Moteur d'Allocation de la Demande}\\
\large{Analyse Technique du Système de Marché FoodOps}}
\author{Documentation Technique}
\date{\today}

\begin{document}

\maketitle

\tableofcontents
\newpage

\section{Introduction}

Le moteur d'allocation de la demande de FoodOps simule de manière réaliste comment les clients choisissent leurs restaurants. Ce système sophistiqué reproduit les mécanismes économiques et comportementaux du marché de la restauration.

\subsection{Principe de fonctionnement}

Imaginez une ville avec plusieurs restaurants et différents types de clients (étudiants, familles, touristes...). Chaque mois, ces clients doivent choisir où manger en fonction de :
\begin{itemize}
    \item Leur budget disponible
    \item L'attractivité du restaurant (qualité, visibilité, menu...)
    \item La disponibilité des places
    \item La concurrence entre restaurants similaires
\end{itemize}

Le système traite ces choix de manière ordonnée et logique, en simulant le comportement réel des consommateurs.

\section{Vue d'ensemble du processus}

\subsection{Le parcours d'un client type}

Prenons l'exemple de Marc, un étudiant avec 12€ de budget mensuel pour les restaurants :

\begin{enumerate}
    \item \textbf{Filtrage initial} : Marc ne considère que les restaurants dont le prix médian ne dépasse pas 14.40€ (son budget + 20\% de tolérance)
    \item \textbf{Évaluation} : Parmi les restaurants accessibles, Marc calcule un "score d'attractivité" pour chacun
    \item \textbf{Choix} : Marc se dirige vers le restaurant avec le meilleur score
    \item \textbf{Disponibilité} : Si ce restaurant est complet, Marc essaie le second meilleur, puis le troisième, etc.
    \item \textbf{Résultat} : Marc mange soit dans un restaurant disponible, soit nulle part s'ils sont tous complets
\end{enumerate}

Ce processus se répète pour tous les clients de tous les segments.

\section{Architecture du système}

\subsection{Composants principaux}

\begin{figure}[H]
\centering
\begin{tikzpicture}[node distance=2cm]
\node (scenario) [rectangle, draw, fill=blue!20] {Scénario (Population + Segments)};
\node (restaurants) [rectangle, draw, fill=green!20, right of=scenario, xshift=2cm] {Liste des Restaurants};
\node (demande) [rectangle, draw, fill=yellow!20, below of=scenario] {Calcul de la Demande};
\node (classement) [rectangle, draw, fill=orange!20, below of=restaurants] {Classement par Attractivité};
\node (allocation) [rectangle, draw, fill=red!20, below of=demande, xshift=1cm] {Allocation Gloutonne};

\draw[->] (scenario) -- (demande);
\draw[->] (restaurants) -- (classement);
\draw[->] (demande) -- (allocation);
\draw[->] (classement) -- (allocation);
\end{tikzpicture}
\caption{Architecture générale du moteur d'allocation}
\end{figure}

\section{Paramètres du marché}

\subsection{Vitesses de service}

Chaque type de restaurant a une efficacité différente dans l'utilisation de sa capacité :

$$\text{Capacité exploitable} = \text{Capacité physique} \times \text{Coefficient de vitesse}$$

\begin{table}[H]
\centering
\begin{tabular}{lcc}
\toprule
Type de restaurant & Coefficient & Explication \\
\midrule
Fast-food & 1.00 & Service ultra-rapide, rotation maximale \\
Bistro & 0.80 & Service à table, tempo modéré \\
Gastro & 0.50 & Repas long, expérience raffinée \\
\bottomrule
\end{tabular}
\caption{Coefficients de vitesse par type de restaurant}
\end{table}

\textbf{Exemple concret} : Un fast-food de 30 places peut théoriquement servir 1800 clients par mois (30 × 2 services × 30 jours), tandis qu'un restaurant gastronomique de même taille ne servira que 900 clients.

\subsection{Tolérance budgétaire}

Les clients acceptent de dépenser jusqu'à 20\% au-dessus de leur budget habituel pour un restaurant particulièrement attractif :

$$\text{Prix maximum accepté} = \text{Budget segment} \times 1.20$$

\section{Calcul de la capacité exploitable}

\subsection{Méthode de calcul}

La capacité exploitable mensuelle combine plusieurs facteurs réalistes :

\begin{algorithm}[H]
\caption{Calcul de la capacité exploitable}
\begin{algorithmic}[1]
\REQUIRE Restaurant avec capacité physique et type
\ENSURE Nombre de clients servables par mois

\STATE $capacite\_base \leftarrow places \times 2 \times 30$
\COMMENT{2 services/jour, 30 jours/mois}

\STATE $coefficient \leftarrow$ SERVICE\_SPEED[type\_restaurant]

\STATE $capacite\_exploitable \leftarrow capacite\_base \times coefficient$

\RETURN $max(0, capacite\_exploitable)$
\end{algorithmic}
\end{algorithm}

\textbf{Intuition} : Cette formule reflète la réalité économique. Un fast-food maximise sa rotation client, tandis qu'un restaurant gastronomique privilégie l'expérience au détriment du volume.

\section{Système de filtrage budgétaire}

\subsection{Processus de sélection}

Avant même de considérer l'attractivité, chaque client élimine les restaurants trop chers :

\begin{algorithm}[H]
\caption{Test d'éligibilité budgétaire}
\begin{algorithmic}[1]
\REQUIRE Restaurant, Segment de clientèle
\ENSURE Vrai si le restaurant est abordable

\STATE $prix\_items \leftarrow$ extraire\_prix\_menu(restaurant)

\IF{$prix\_items$ est vide}
    \STATE $prix\_median \leftarrow 0$
\ELSE
    \STATE $prix\_median \leftarrow$ médiane($prix\_items$)
\ENDIF

\STATE $budget\_segment \leftarrow$ BUDGET\_PER\_SEGMENT[segment]

\STATE $seuil\_tolerance \leftarrow budget\_segment \times 1.20$

\RETURN $prix\_median \leq seuil\_tolerance$
\end{algorithmic}
\end{algorithm}

\textbf{Exemple pratique} :
\begin{itemize}
    \item Un étudiant (budget 12€) acceptera un restaurant avec un prix médian jusqu'à 14.40€
    \item Mais rejettera automatiquement un restaurant avec un prix médian de 18€
\end{itemize}

\section{Gestion de la cannibalisation}

\subsection{Principe de la concurrence}

Quand plusieurs restaurants du même type coexistent, ils se disputent la même clientèle. Cette concurrence réduit l'efficacité de chacun.

\subsection{Calcul du facteur de pénalité}

\begin{algorithm}[H]
\caption{Facteur de cannibalisation}
\begin{algorithmic}[1]
\REQUIRE Restaurant, Comptage des types sur le marché
\ENSURE Facteur de réduction du score (entre 0 et 1)

\STATE $n \leftarrow$ nombre\_restaurants\_meme\_type

\IF{$n \leq 1$}
    \RETURN $1.0$ \COMMENT{Aucune concurrence}
\ENDIF

\STATE $penalite \leftarrow \sqrt{1 + 0.5 \times (n-1)}$

\RETURN $\frac{1}{penalite}$
\end{algorithmic}
\end{algorithm}

\textbf{Impact concret} :
\begin{itemize}
    \item 1 seul fast-food : facteur = 1.0 (aucune pénalité)
    \item 2 fast-foods : facteur ≈ 0.82 (réduction de 18\%)
    \item 3 fast-foods : facteur ≈ 0.71 (réduction de 29\%)
    \item 4 fast-foods : facteur ≈ 0.63 (réduction de 37\%)
\end{itemize}

Cette formule reproduit l'observation économique que la concurrence directe dilue l'efficacité de chaque acteur.

\section{Classement par attractivité}

\subsection{Construction du palmarès}

Pour chaque segment de clientèle, le système établit un classement des restaurants les plus attractifs :

\begin{algorithm}[H]
\caption{Classement des restaurants par segment}
\begin{algorithmic}[1]
\REQUIRE Liste des restaurants, Segment cible, Comptage des types
\ENSURE Liste triée (index\_restaurant, score\_final)

\STATE $classement \leftarrow$ liste\_vide()

\FOR{chaque restaurant $R_i$ dans la liste}
    \IF{PAS eligible\_budget($R_i$, segment)}
        \STATE continuer \COMMENT{Restaurant trop cher, ignoré}
    \ENDIF

    \STATE $score\_base \leftarrow$ attraction\_score($R_i$, segment)
    \STATE $facteur\_canni \leftarrow$ cannibalisation($R_i$)
    \STATE $score\_final \leftarrow score\_base \times facteur\_canni$

    \STATE ajouter($classement$, ($i$, $score\_final$))
\ENDFOR

\STATE trier($classement$, par score décroissant)

\RETURN $classement$
\end{algorithmic}
\end{algorithm}

\textbf{Exemple de résultat} :
\begin{center}
\begin{tabular}{lcc}
\toprule
Restaurant & Score final & Rang \\
\midrule
Burger King \#1 & 8.2 & 1er \\
Pizza Corner & 7.1 & 2ème \\
Burger King \#2 & 6.7 & 3ème \\
Bistro du coin & 4.3 & 4ème \\
\bottomrule
\end{tabular}
\end{center}

\section{Algorithme d'allocation principal}

\subsection{Stratégie gloutonne}

L'allocation suit une approche "glouton optimal" : on satisfait d'abord les clients avec leur premier choix, puis on gère les débordements.

\begin{algorithm}[H]
\caption{Allocation de la demande - Vue d'ensemble}
\begin{algorithmic}[1]
\REQUIRE Liste des restaurants, Scénario de marché
\ENSURE Dictionnaire allocation[restaurant] = nombre\_clients

\STATE \textbf{Phase 1 : Préparation}
\STATE $demande \leftarrow$ calculer\_demande\_par\_segment(scénario)
\STATE $comptage\_types \leftarrow$ compter\_restaurants\_par\_type()
\STATE $capacités \leftarrow$ calculer\_capacités\_exploitables()
\STATE $allocation \leftarrow$ initialiser\_à\_zéro()

\STATE \textbf{Phase 2 : Traitement par segment}
\FOR{chaque segment avec $demande[segment] > 0$}
    \STATE $classement \leftarrow$ classer\_restaurants(segment)
    \STATE allouer\_segment\_glouton(segment, classement, capacités, allocation)
\ENDFOR

\RETURN $allocation$
\end{algorithmic}
\end{algorithm}

\subsection{Allocation détaillée par segment}

\begin{algorithm}[H]
\caption{Allocation gloutonne pour un segment}
\begin{algorithmic}[1]
\REQUIRE Segment, Classement des restaurants, Capacités, Allocation courante
\ENSURE Mise à jour de l'allocation

\STATE $clients\_restants \leftarrow demande[segment]$

\FOR{chaque restaurant $R_i$ dans le classement (du meilleur au pire)}
    \IF{$clients\_restants = 0$}
        \STATE sortir \COMMENT{Tous les clients sont placés}
    \ENDIF

    \IF{$capacité[R_i] = 0$}
        \STATE continuer \COMMENT{Restaurant complet}
    \ENDIF

    \STATE $places\_dispo \leftarrow capacité[R_i]$
    \STATE $à\_placer \leftarrow \min(clients\_restants, places\_dispo)$

    \STATE $allocation[R_i] \leftarrow allocation[R_i] + à\_placer$
    \STATE $capacité[R_i] \leftarrow capacité[R_i] - à\_placer$
    \STATE $clients\_restants \leftarrow clients\_restants - à\_placer$
\ENDFOR

\IF{$clients\_restants > 0$}
    \STATE \textbf{Note} : Ces clients sont "perdus" (aucun restaurant disponible)
\ENDIF
\end{algorithmic}
\end{algorithm}

\subsection{Exemple d'exécution}

Supposons 3 restaurants et 100 étudiants :

\begin{center}
\begin{tabular}{lccc}
\toprule
Restaurant & Score & Capacité & Allocation finale \\
\midrule
Fast-food A & 8.5 & 60 & 60 étudiants \\
Bistro B & 6.2 & 25 & 25 étudiants \\
Fast-food C & 4.1 & 30 & 15 étudiants \\
\midrule
\textbf{Total} & & \textbf{115} & \textbf{100 étudiants} \\
\bottomrule
\end{tabular}
\end{center}

\textbf{Déroulé} :
\begin{enumerate}
    \item Les 60 premiers étudiants vont au Fast-food A (meilleur score)
    \item Les 25 suivants vont au Bistro B (Fast-food A complet)
    \item Les 15 derniers vont au Fast-food C
    \item Aucun client perdu dans cet exemple
\end{enumerate}

\section{Conversion de la demande}

\subsection{Du scénario aux volumes réels}

Le scénario définit des proportions relatives que le système convertit en nombres absolus :

\begin{algorithm}[H]
\caption{Calcul des volumes par segment}
\begin{algorithmic}[1]
\REQUIRE Scénario (population totale + parts par segment)
\ENSURE Dictionnaire demande[segment] = nombre\_clients

\STATE $demande \leftarrow$ dictionnaire\_vide()

\FOR{chaque segment dans le scénario}
    \STATE $part \leftarrow$ scénario.parts[segment]
    \STATE $volume \leftarrow$ arrondir(population\_totale $\times$ part)
    \STATE $demande$[segment] $\leftarrow$ volume
\ENDFOR

\RETURN $demande$
\end{algorithmic}
\end{algorithm}

\textbf{Exemple} : Population de 5000 habitants
\begin{itemize}
    \item Étudiants (60\%) → 3000 clients
    \item Actifs (25\%) → 1250 clients
    \item Familles (10\%) → 500 clients
    \item Touristes (5\%) → 250 clients
\end{itemize}

\section{Mécanisme de sécurité}

\subsection{Validation finale des capacités}

Une fonction de sécurité garantit que l'allocation respecte les contraintes physiques :

\begin{algorithm}[H]
\caption{Validation des capacités}
\begin{algorithmic}[1]
\REQUIRE Restaurants, Allocation proposée
\ENSURE Allocation corrigée et valide

\STATE $allocation\_finale \leftarrow$ dictionnaire\_vide()

\FOR{chaque restaurant $R_i$}
    \STATE $capacité\_max \leftarrow$ capacité\_exploitable($R_i$)
    \STATE $allocation\_proposée \leftarrow$ allocation[$R_i$]
    \STATE $allocation\_finale[R_i] \leftarrow \min(allocation\_proposée, capacité\_max)$
\ENDFOR

\RETURN $allocation\_finale$
\end{algorithmic}
\end{algorithm}

Cette étape protège contre d'éventuelles incohérences dues à des modifications externes du système.

\section{Analyse de performance}

\subsection{Complexité algorithmique}

\textbf{Complexité temporelle} : $O(S \times R \times \log R)$
\begin{itemize}
    \item $S$ = nombre de segments de clientèle
    \item $R$ = nombre de restaurants
    \item Le terme $\log R$ vient du tri par attractivité
\end{itemize}

\textbf{Complexité spatiale} : $O(R)$
\begin{itemize}
    \item Stockage des capacités et allocations par restaurant
    \item Indépendant du nombre de clients total
\end{itemize}

\textbf{Performance pratique} :
\begin{itemize}
    \item 10 restaurants, 5 segments → ~200 opérations
    \item 100 restaurants, 10 segments → ~7000 opérations
    \item Très efficace même pour de grandes simulations
\end{itemize}

\section{Propriétés du système}

\subsection{Invariants garantis}

\begin{enumerate}
    \item \textbf{Respect des capacités} : Aucun restaurant ne reçoit plus de clients qu'il ne peut en servir
    \item \textbf{Conservation de la demande} : La somme des allocations ne dépasse jamais la demande totale
    \item \textbf{Cohérence des préférences} : Les clients vont toujours vers leur choix disponible le plus attractif
    \item \textbf{Déterminisme} : À données identiques, le résultat est toujours le même
\end{enumerate}

\subsection{Comportements émergents}

\begin{itemize}
    \item \textbf{Effet de saturation} : Les restaurants populaires se remplissent en premier
    \item \textbf{Redistribution naturelle} : La clientèle se reporte automatiquement vers les alternatives
    \item \textbf{Équilibrage concurrentiel} : La cannibalisation encourage la différenciation
    \item \textbf{Segmentation du marché} : Chaque type de restaurant trouve sa clientèle naturelle
\end{itemize}

\section{Cas d'usage et exemples}

\subsection{Scénario de test complet}

\textbf{Configuration} :
\begin{itemize}
    \item 4 restaurants : 2 fast-foods (40 places), 1 bistro (25 places), 1 gastro (15 places)
    \item 2000 clients : 70\% étudiants (budget 12€), 30\% cadres (budget 30€)
\end{itemize}

\textbf{Étapes de calcul} :

\textit{1. Capacités exploitables :}
\begin{itemize}
    \item Fast-food 1 : $40 \times 2 \times 30 \times 1.0 = 2400$ clients/mois
    \item Fast-food 2 : $40 \times 2 \times 30 \times 1.0 = 2400$ clients/mois
    \item Bistro : $25 \times 2 \times 30 \times 0.8 = 1200$ clients/mois
    \item Gastro : $15 \times 2 \times 30 \times 0.5 = 450$ clients/mois
\end{itemize}

\textit{2. Demande par segment :}
\begin{itemize}
    \item Étudiants : $2000 \times 0.7 = 1400$ clients
    \item Cadres : $2000 \times 0.3 = 600$ clients
\end{itemize}

\textit{3. Facteurs de cannibalisation :}
\begin{itemize}
    \item Fast-foods : $1/\sqrt{1.5} \approx 0.82$
    \item Bistro et Gastro : $1.0$ (uniques)
\end{itemize}

\textit{4. Résultat probable :}
\begin{itemize}
    \item Les étudiants se concentrent sur les fast-foods (prix abordable)
    \item Les cadres se répartissent entre bistro et gastro (budget plus large)
    \item Utilisation optimale des capacités disponibles
\end{itemize}

\section{Avantages et limitations}

\subsection{Forces du système}

\begin{itemize}
    \item \textbf{Réalisme économique} : Reproduit fidèlement les mécanismes de marché
    \item \textbf{Modularité} : Chaque composant peut être modifié indépendamment
    \item \textbf{Transparence} : Chaque étape est explicite et vérifiable
    \item \textbf{Robustesse} : Gestion élégante des cas limites
    \item \textbf{Efficacité} : Performance adaptée aux simulations en temps réel
\end{itemize}

\subsection{Limitations identifiées}

\begin{itemize}
    \item \textbf{Modèle budgétaire simplifié} : Prix médian comme seul critère
    \item \textbf{Statisme temporal} : Pas d'évolution des préférences dans le temps
    \item \textbf{Cannibalisation uniforme} : Même impact pour tous les restaurants d'un type
    \item \textbf{Absence de géographie} : Pas de notion de distance ou de zones
\end{itemize}

\section{Extensions possibles}

\subsection{Améliorations à court terme}

\begin{enumerate}
    \item \textbf{Modèle budgétaire avancé}
    \begin{itemize}
        \item Considérer la variance des prix (pas seulement la médiane)
        \item Introduire des courbes d'élasticité prix-demande
    \end{itemize}

    \item \textbf{Facteurs géographiques}
    \begin{itemize}
        \item Distance client-restaurant
        \item Zones de chalandise
        \item Coûts de transport
    \end{itemize}
\end{enumerate}

\subsection{Évolutions à long terme}

\begin{enumerate}
    \item \textbf{Dynamiques temporelles}
    \begin{itemize}
        \item Fidélisation de la clientèle
        \item Effets de mode et saisonnalité
        \item Apprentissage des préférences
    \end{itemize}

    \item \textbf{Complexité concurrentielle}
    \begin{itemize}
        \item Cannibalisation géographique
        \item Alliances entre restaurants
        \item Stratégies de différenciation
    \end{itemize}
\end{enumerate}

\section{Conclusion}

Le moteur d'allocation de FoodOps constitue un système élégant qui transforme des concepts économiques complexes en algorithmes clairs et efficaces. Sa conception modulaire facilite la maintenance et l'évolution, tandis que ses mécanismes reproduisent fidèlement les dynamiques observées sur les marchés de restauration réels.

L'approche gloutonne, bien que simple conceptuellement, produit des résultats cohérents avec les théories économiques de concurrence et de segmentation de marché. Les mécanismes de cannibalisation et de filtrage budgétaire ajoutent le réalisme nécessaire pour des simulations crédibles.

Ce système forme une base solide pour des analyses de marché plus sophistiquées et peut facilement s'adapter aux besoins spécifiques de différents contextes de simulation.

\end{document}
